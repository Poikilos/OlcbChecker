\documentclass[11pt]{article}
\usepackage{geometry}                % See geometry.pdf to learn the layout options. There are lots.
\geometry{letterpaper}                   % ... or a4paper or a5paper or ... 
%\geometry{landscape}                % Activate for for rotated page geometry
\usepackage[parfill]{parskip}    % Activate to begin paragraphs with an empty line rather than an indent
\usepackage{graphicx}
\usepackage{amssymb}
\usepackage{epstopdf}
\usepackage{hyperref}

\DeclareGraphicsRule{.tif}{png}{.png}{`convert #1 `dirname #1`/`basename #1 .tif`.png}

% macro for inserting common introduction text
% takes hyperlink to a stadnard and the standard name as arguments
\newcommand{\introductionCaveats}[2]{

    \section{Introduction}

    This note documents the procedure for checking an OpenLCB implementation against the 
    \href{#1}
    {#2}.

    The checks are traceable to specific sections of the Standard.

    The checking assumes that the Device Being Checked (DBC) is being exercised by other
    nodes on the message network, 
    e.g. is responding to enquiries from other parts of the message network.
}

% macro for inserting the procedure section
\newcommand{\checkProcedure}[1]{

    Select ``#1" in the program, 
    then select each section below in turn.  Follow the prompts
    for when to reset/restart the node and when to check 
    outputs against the node documentation.
}



\title{Installing the OpenLCB Checker Software\linebreak{}Basic Version}

\begin{document}
\maketitle

\section{Introduction}

This document describes how to obtain and run a set of basic checks for 
OpenLCB node implementations.  

The checks are based on the Python `openlcb' module.
More information on that can be obtained from its
\href{https://github.com/bobjacobsen/PythonOlcbNode}{GitHub project site.}

For more information on the checks, see the
\href{https://github.com/bobjacobsen/OlcbChecker/tree/main/plans/}{directory of checking plans}.

\section{Obtaining the Software}

The checker software is distributed as a set of inter-connected Python source files.

\subsection{Obtaining and Using via Git}

If you're using Git, 
\begin{verbatim}
cd (where you want to put this)
git clone https://github.com/bobjacobsen/OlcbChecker.git
\end{verbatim}
will create a OlcbChecker directory containing the most recent version of the software.
This also contains git tags for the released versions.

\subsection{Obtaining by Downloading a .zip File}

You can get a download of the most recent released version by going to the project's 
\href{https://github.com/bobjacobsen/OlcbChecker/tags}{Github releases web page tag section}
\footnote{Linked above or see \href{https://github.com/bobjacobsen/OlcbChecker/tags}{https://github.com/bobjacobsen/OlcbChecker/tags}}
and clicking the .zip or .tgz icon on the most recent release.

To get the very most recent version,
\footnote{But if you want to stay current with development of the tools, you should probably be using Git.}
go to the project's
\href{https://github.com/bobjacobsen/OlcbChecker}{Github main web page tag section},
click the green Code button, and select "Download Zip".

Expand the downloaded file in a suitable place.

\subsection{Prerequisites}

You need to have Python 3.10 installed to run the program. Consult your
computer's documentation for how to install that.  Many computers already
have it installed.

You have to manually install the `openlcb` module by checking out the 
PythonOlcbNode repository from GitHub.
\footnote{This will eventually be available via PIP, but not yet.}
To do this:
\begin{verbatim}
cd (where you want to put it)
git clone https://github.com/bobjacobsen/PythonOlcbNode.git
python3.10 -m pip install --editable PythonOlcbNode
\end{verbatim}
This will create a PythonOlcbNode directory containing the most recent version of the software
and make that available as a `openlcb` Python module.

To run the CDI checks, the `xmlschema' Python module must be installed. To do that, 
enter\footnote{This is the command for Linux nad MacOS; the Windows command may be different.}

\begin{verbatim}
python3.10 -m pip install xmlschema
\end{verbatim}

\section{Starting the Program}

First, do
\begin{verbatim}
cd (your OlcbChecker directory)
\end{verbatim}
to get to the right directory for running the code. 

To start the program:
\begin{verbatim}
python3.10 control_master.py
\end{verbatim}

Depending on your Python installation, this simpler form may also work:
\begin{verbatim}
./control_master.py
\end{verbatim}


\section{Configuring the Checker Program}

When you first start the program, you'll be shown a basic menu:

\begin{verbatim}
OpenLCB checking program
 s Setup

 0 Frame Transport checking
 1 Message Network checking
 2 SNIP checking
 3 Event Transport checking
 4 Datagram Transport checking
 5 Memory Configuration checking
 6 CDI checking
  
 q  Quit
>> 
\end{verbatim}

Type s and hit return to get the setup menu:

\begin{verbatim}
The current settings are:
  hostname = None
  portnumber = 12021
  devicename = None
  targetnodeid = None
  ownnodeid = 03.00.00.00.00.01
  checkpip = True
  trace = 10

c change setting
h help
r return
>> 
\end{verbatim}

At a minimum, you should define how to connect to your OpenLCB network,
and the Node ID of the device you want to check.  

To change the Node ID, select the``change setting" option and work through the prompts:

\begin{verbatim}
>> c
enter variable name
>> targetnodeid
enter new value
>> 02.01.57.00.04.9A
The current settings are:
  hostname = None
  portnumber = 12021
  devicename = None
  targetnodeid = 02.01.57.00.04.9A
  ownnodeid = 03.00.00.00.00.01
  checkpip = True
  trace = 10

c change setting
h help
r return

>> 
\end{verbatim}

Get the proper value from either a label on the device, or from its documentation.
\footnote{Some checks, but not all, can determine the node ID themselves if you leave
    the value as None. This is only reliable if there's just one node on your OpenLCB
    network.  Note that some OpenLCB hubs add a node of their own to the node
    being checked.}
    
There are currently two ways to connect the program to your OpenLCB network:
\begin{enumerate}
\item Via a USB-CAN adapter, or
\item Via a GridConnect-format TCP/IP connection.
\end{enumerate}

For a USB-CAN connector, define the devicename to be the address of the device in your computer, 
e.g. /dev/cu.usbmodemCC570001B1 or COM7.

For a TCP/IP link, define the hostname to be the IP address or host name to be used 
for connecting.

You must specify one or the other of hostname and device name, but not both.
When you enter one, the other will be set to None.

When done with setup, select r for return.  

You'll be asked if you want to save changes.  
Select y to save and n to skip saving.

\begin{verbatim}
>> r

Do you want to save the new settings? (y/n)
 >> y
Stored
Quit and restart the program to put them into effect

OpenLCB checking program
 s Setup

 0 Frame Transport checking
 1 Message Network checking
 2 SNIP checking
 3 Event Transport checking
 4 Datagram Transport checking
 5 Memory Configuration checking
 6 CDI checking
  
 q  Quit
>> 
\end{verbatim}

Quit and restart the program to put your changes into effect.

\section{Running Checks}

\subsection{Required Equipment}

It's generally best to have the device being checked (DBC) 
as the only device on the OpenLCB network. 

If a direct CAN connection will be used,
a supported USB-CAN adapter is required
\footnote{The checker has been checked with the
\href{https://www.rr-cirkits.com/description/LCC-usb-flyer.pdf}{RR-CirKits LCC buffer-USB},
but others with similar operational characteristics will probably work.
}. 
Connect the adapter to your computer as indicated in its instructions.
Connect the adapter to the DBC using a single UTP cable
and attach two CAN terminators.

If a TCP/IP GridConnect connection will be used, 
configure the DBC to connect to the TCP/IP hub when restarted. Note that if 
the DBC is providing the hub for the connection and restarting the DBC
breaks connections to that hub, several of the checks will indicate problems
due to the connection breaking.

Provide power to the DBC using its recommended equipment and connections.

\subsection{The Checking Sequence}

The checks are categorized by the Standard they primarily check:

\begin{verbatim}
OpenLCB checking program
 s Setup

 0 Frame Transport checking
 1 Message Network checking
 2 SNIP checking
 3 Event Transport checking
 4 Datagram Transport checking
 5 Memory Configuration checking
 6 CDI checking
\end{verbatim}

It's generally best to run them in order, as issues flagged by earlier
checks may prevent later ones from running properly.

Once you select a section, you'll be presented with a sub-menu of the
actual checks.  These match the sections in the relevant plan document.
For example:

\begin{verbatim}
Frame Transport Standard checking
 a Run all in sequence
 1 Initialization checking
 2 AME checking
 3 Collision checking
 4 reserved bit checking
\end{verbatim}

"Run all in sequence" is usually a good starting point.

The checker may prompt you to reset/restart the node or to 
compare something to the node documentation. For example, 
the Initialization checking requests that you restart the node
so that its start-up sequence can be checked for validity.
It waits for about 30 seconds, and if it still hasn't seen anything, the
check fails.

\section{Technical Information}

Your selected defaults are stored in the localoverrides.py file.
The original values are stored in the defaults.py file.

Should something corrupt the localoverrides.py file,
you can delete it, restart the program, and re-enter your configuration.

There are several kinds of Python scripts in the package. 
\begin{enumerate}
\item Scripts named check\_*.py, which hold the checks for each specific step.
    The naming contention is e.g. check\_fr20\_ame.py which is the 2nd (20)
    check of the frame level (fr) and somes some AME checking (ame).
    \footnote{The numbers are spaced by 10 to allow us to add new scripts in between.}
    These can be individually run e.g. to rapidly exercise a feature for debugging.
\item Scripts name control\_*.py which hold the simple keyboard-oriented 
    program for running the check scripts.  control\_master.py is the 
    main script, which invokes a control\_*.py script for each level of the 
    checking and setup.
\item The olcbchecker directory and package contains general implementation 
    files.
\end{enumerate}



\end{document}  
