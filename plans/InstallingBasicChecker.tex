\documentclass[11pt]{article}
\usepackage{geometry}                % See geometry.pdf to learn the layout options. There are lots.
\geometry{letterpaper}                   % ... or a4paper or a5paper or ... 
%\geometry{landscape}                % Activate for for rotated page geometry
\usepackage[parfill]{parskip}    % Activate to begin paragraphs with an empty line rather than an indent
\usepackage{graphicx}
\usepackage{amssymb}
\usepackage{epstopdf}
\usepackage{hyperref}

\DeclareGraphicsRule{.tif}{png}{.png}{`convert #1 `dirname #1`/`basename #1 .tif`.png}

\title{Installing the OpenLCB Checker Software\linebreak{}Basic Version}
\author{The OpenLCB Group}
%\date{}                                         % Activate to display a given date or no date

\begin{document}
\maketitle


\section{Introduction}

This document describes how to obtain and run a set of basic checks for 
OpenLCB nodes.  

The checks are based on the Python `openlch' module.
More information on that can be obtained from its
\href{https://github.com/bobjacobsen/PythonOlcbNode}{GitHub project site.}

For more information on the checks, see the
\href{https://github.com/bobjacobsen/OlcbChecker/blob/main/README.md}{package documentation}
or the 
\href{https://github.com/bobjacobsen/OlcbChecker/tree/main/plans/}{directory of checking plans}.

\section{Obtaining the Software}

The software is distributed as a set of inter-connected Python source files.

\subsection{Obtaining and Using via Git}

If you're using Git, 
\begin{verbatim}
cd (where you want to put this)
git clone https://github.com/bobjacobsen/OlcbChecker.git
\end{verbatim}
will create a OlcbChecker directory containing the most recent version of the software.
This also contains git tags for the released versions.

\subsection{Obtaining by Downloading a .zip File}

You can get a download of the most recent released version by going to the project's 
\href{https://github.com/bobjacobsen/OlcbChecker/tags}{Github releases web page tag section}
\footnote{Linked above or see \href{https://github.com/bobjacobsen/OlcbChecker/tags}{https://github.com/bobjacobsen/OlcbChecker/tags}}
and clicking the .zip or .tgz icon on the most recent release.

To get the very most recent version,
\footnote{But if you want to stay current with development of the tools, you should probably be using Git.}
go to the project's
\href{https://github.com/bobjacobsen/OlcbChecker}{Github main web page tag section},
click the green Code button, and select "Download Zip".

Expand the downloaded file in a suitable place.

\subsection{Prerequisites}

You need to have Python 3.10 installed to run the program. Consult your
computer's documentation for how to install that.  Many computers already
have it installed.

You have to manually install the `openlcb` module by checking out the 
PythonOlcbNode repository from GitHub.
\footnote{This will eventually be available via PIP, but not yet.}
To do this:
\begin{verbatim}
cd (where you want to put it)
git clone https://github.com/bobjacobsen/PythonOlcbNode.git
\end{verbatim}
will create a PythonOlcbNode directory containing the most recent version of the software.

To run the CDI checks, the `xmlschema' Python module must be installed. To do that, 
enter\footnote{This is the command for Linux nad MacOS; the Windows command may be different.}

\begin{verbatim}
python3 -m pip install xmlschema
\end{verbatim}

\section{Configuring for Running}

You need to have PYTHONPATH defined to include the PythonOlcbNode directory.
\footnote{Eventually, this will no longer be necessary, but not quite yet.}
In the Linux and macOS
terminals, you can do this with

\begin{verbatim}
cd (where you put the PythonOlcbNode directory)
export PYTHONPATH=$PWD
\end{verbatim}

or you can place the equivalent line in your shell configuration file.

If you don't add this to your shell configuration file, you'll have to do 
this each time you start a terminal session.

Next
\begin{verbatim}
cd (where you put the OlcbChecker directory)
\end{verbatim}
to get to the right directory for running the code. 

To start the program:
\begin{verbatim}
python3.10 control_master.py
\end{verbatim}

Depending on your Python installation, this simpler form may also work:
\begin{verbatim}
./control_master.py
\end{verbatim}


\section{Configuring the Checker Program}

When you first start the program, you'll be shown a basic menu:

\begin{verbatim}
OpenLCB checking program
 s Setup

 0 Frame Transport checking
 1 Message Network checking
 2 SNIP checking
 3 Event Transport checking
 4 Datagram Transport checking
 5 Memory Configuration checking
  
 q  Quit
>> 
\end{verbatim}

Type 0 and hit return to get the setup menu:

\begin{verbatim}
The current settings are:
  hostname = None
  portnumber = 12021
  devicename = None
  targetnodeid = None
  ownnodeid = 03.00.00.00.00.01
  checkpip = True
  trace = 10

c change setting
h help
r return
>> 
\end{verbatim}

At a minimum, you should define how to connect to your OpenLCB network,
and the Node ID of the device you want to check.  

To change the Node ID, select the``change setting" option and work through the prompts:

\begin{verbatim}
>> c
enter variable name
>> targetnodeid
enter new value
>> 02.01.57.00.04.9A
The current settings are:
  hostname = None
  portnumber = 12021
  devicename = None
  targetnodeid = 02.01.57.00.04.9A
  ownnodeid = 03.00.00.00.00.01
  checkpip = True
  trace = 10

c change setting
h help
r return

>> 
\end{verbatim}

Get the proper value from either a label on the device, or from its documentation.
\footnote{Some checks, but not all, can determine the node ID themselves if you leave
    the value as None. This is only reliable if there's just one node on your OpenLCB
    network.  Note that some OpenLCB hubs add a node of their own to the node
    being checked.}
    
There are currently two ways to connect the program to your OpenLCB network:
\begin{enumerate}
\item Via a USB-CAN adapter, or
\item Via a GridConnect-format TCP/IP connection.
\end{enumerate}

For a USB-CAN connector, define the devicename to be the address of the device in your computer, 
e.g. /dev/cu.usbmodemCC570001B1 or COM7.

For a TCP/IP link, define the hostname to be the IP address or host name to be used 
for connecting.

You must specify one or the other of hostname and device name, but not both.
When you enter one, the other will be set to None.

When done with setup, select return.  You'll be asked if you want to save changes.  
Select y to save and n to skip saving.

\begin{verbatim}
>> r

Do you want to save the new settings? (y/n)
 >> y
Stored
Quit and restart the program to put them into effect

OpenLCB checking program
 s Setup

 0 Frame Transport checking
 1 Message Network checking
 2 SNIP checking
 3 Event Transport checking
 4 Datagram Transport checking
 5 Memory Configuration checking
  
 q  Quit
>> 
\end{verbatim}

Quit and restart the program to put your changes into effect.

\section{Running Checks}

\subsection{Required Equipment}

It's generally best to have the device being checked (DBC) 
as the only device on the OpenLCB network. 

If a direct CAN connection will be used,
a supported USB-CAN adapter is required
\footnote{The checker has been checked with the
\href{https://www.rr-cirkits.com/description/LCC-usb-flyer.pdf}{RR-CirKits LCC buffer-USB},
but others with similar operational characteristics will probably work.
}. 
Connect the adapter to your computer as indicated in its instructions.
Connect the adapter to the DBC using a single UTP cable
and attach two CAN terminators.

If a TCP/IP GridConnect connection will be used, 
configure the DBC to connect to the TCP/IP hub when restarted. Note that if 
the DBC is providing the hub for the connection and restarting the DBC
breaks connections to that hub, several restarting checks will indicate problems
when the connection breaks.

Provide power to the DBC using its recommended equipment and connections.


\section{Technical Information}

Your selected defaults are stored in the localoverrides.py file.
The original values are stored in the defaults.py file.

Should something corrupt the localoverrides.py file,
you can delete it, restart the program, and re-enter your configuration.


\end{document}  
